% !TeX root = ./Thesis.tex
% -- TemplateKnob

% -----------------------------------------------------------------------
%   >> Personal Information to be filled in by thesis author !!!   <<
% -----------------------------------------------------------------------
\newcommand{\myTitle}{HiPS: Steganography in chat messages using local large language models on Android}
\newcommand{\myDegree}{Bachelor's Thesis}
\newcommand{\myVersion}{0.1}
\newcommand{\myName}{Tobias Vonderheidt}
\newcommand{\mySupervisor}{M.Sc. Nils Rollshausen}
\newcommand{\myThesiscode}{SEEMOO-BSC-$0368$}           % You will get this from our secretary
\newcommand{\myTime}{\formatdate{09}{05}{2025}}         % hand-in date of the thesis
\newcommand{\myAbstract}{While you read this sentence, around one million WhatsApp messages are being sent. We share the details of our personal lives in them, from our highest hopes to our deepest fears. This is only possible because we assume them to be private. To protect user privacy, we implement an Android app called Hiding in Plain Sight (HiPS). It creates a layer of protection currently missing from popular instant messengers: Steganography. We run large language models (LLMs) locally on smartphones to encode secret messages into inconspicuous cover texts. Our work is based on the Stegasuras project. By porting it to llama.cpp, we are able to swap LLMs to scale with available hardware. We create chat conversations between cover texts, fine-tune their quality via a system prompt and plain text messages, and enrich them with emojis. We conduct a survey amongst students of TU Darmstadt to compare our cover texts to real chat conversations. This delivers valuable insights in how to fine-tune cover text quality for this highly relevant demographic of users.} % at the very end, put "clean" abstract here
\newcommand{\myYearPublication}{2025}                   % year of publication (for copyright and footer)


% -----------------------------------------------------------------------
% General Information of SEEMOO and the TU Darmstadt.
% -----------------------------------------------------------------------
\newcommand{\myFaculty}{Department of Computer Science}
\newcommand{\myFacultyDE}{Fachbereich Informatik}
\newcommand{\myDepartment}{Secure Mobile Networking Lab}
\newcommand{\myDepartmentDE}{Fachgebiet Sichere Mobile Netze}
\newcommand{\myUni}{\protect{Technische Universität Darmstadt}}
\newcommand{\myUniKennziffer}{D17}
\newcommand{\myLocation}{Darmstadt}


% -----------------------------------------------------------------------
% The following are only required to be filled in for PhD Theses.
% -----------------------------------------------------------------------
\newcommand{\myURN}{urn:nbn:de:tuda-tuprints-83253}
\newcommand{\myDegreePhD}{Doktor-Ingenieur (Dr.-Ing.)}
\newcommand{\myBirthDate}{\formatdate{1}{1}{1970}}
\newcommand{\myBirthPlace}{Darmstadt, Deutschland}
\newcommand{\myNationality}{German}
\newcommand{\myProf}{Prof. Dr.-Ing. Matthias Hollick}
\newcommand{\myOtherProf}{Put name here}
\newcommand{\myYearPresent}{1337}                                % year of disputation
\newcommand{\myTimePresent}{\formatdate{01}{01}{\myYearPresent}} % date of disputation