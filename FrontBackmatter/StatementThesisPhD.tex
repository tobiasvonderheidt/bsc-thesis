%!TEX root = ../Thesis.tex

\begin{otherlanguage}{ngerman}

%*******************************************************
\chapterExtra{Erklärung zur Dissertationsschrift}
%*******************************************************

% Promotionsordnung und mehr des FB 20
% https://www.informatik.tu-darmstadt.de/forschung_fb20/wissenschaftliche_karriere/promotion/index.de.jsp

\begin{flushright}
    \emph{\small gemäß §\,9 der Allgemeinen Bestimmungen der Promotionsordnung der \\
    Technischen Universität Darmstadt vom \formatdate{12}{1}{1990} (ABI. 1990, S.\,658) \\
    in der Fassung der 8.\,Novelle vom \formatdate{1}{3}{2018}}
\end{flushright}
Hiermit versichere ich, \myName{}, die vorliegende Dissertationsschrift ohne Hilfe Dritter und nur mit den angegebenen Quellen und Hilfsmitteln angefertigt zu haben. Alle Stellen, die Quellen entnommen wurden, sind als solche kenntlich gemacht. Eigenzitate aus vorausgehenden wissenschaftlichen Veröffentlichungen sowie die Urheberschaften der einzelnen Beiträge sind in Anlehnung an die Hinweise des Promotionsausschusses des Fachbereichs Informatik zum Thema \enquote{Kumulative Dissertation und Eigenzitate in Dissertationen} (CR; 01.12.2022) im Kapitel \enquote{\emph{Collaborations and My Contribution}} auf den \cpagerefrange*{ch:Collaborations}{ch:CollaborationsEnd} angegeben. Diese Arbeit hat in gleicher oder ähnlicher Form noch keiner Prüfungsbehörde vorgelegen. In der abgegebenen Dissertationsschrift stimmen die schriftliche und die elektronische Fassung überein.
% cannot use \nameref for chapter name, see
% https://bitbucket.org/amiede/classicthesis/issues/170/nameref-for-chapter-showing-the-previous

\bigskip

\noindent\textit{\myLocation{}, \myTime{}}

\begin{flushright}
    \begin{tabular}{m{5cm}}
        \\ \hline
        \centering\myName{} \\
    \end{tabular}
\end{flushright}

\end{otherlanguage}
