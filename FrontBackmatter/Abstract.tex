% !TeX root = ../Thesis.tex

%*******************************************************
% Abstract
%*******************************************************
\begingroup
\let\clearpage\relax
\let\cleardoublepage\relax
\let\cleardoublepage\relax

\chapterExtra{Abstract}
While you read this sentence, around one million WhatsApp messages are being sent. We share the details of our personal lives in them, from our highest hopes to our deepest fears. This is only possible because we assume them to be private.

To protect user privacy, we implement an Android app called \gls{HiPS}. It creates a layer of protection currently missing from popular instant messengers: Steganography. We run \glspl{LLM} locally on the smartphone to encode secret messages into unsuspicious cover texts. Our work is based on the Stegasuras project. We extend its functionality in multiple ways: By porting it to llama.cpp, we are able to swap \glspl{LLM} to scale with available hardware. We create chat conversations of arbitrarily interleaved cover texts and plain texts, expose a system prompt for easy low-level access to the \gls{LLM} via natural language, and enrich cover texts with emojis to make them more organic.

To compare our cover texts to real chat messages, we conduct a survey amongst students of TU Darmstadt. This delivers valuable insights in how to fine-tune cover text quality for this highly relevant demographic of users.

\vfill

\begin{otherlanguage}{ngerman}
\chapterExtra{Zusammenfassung}
Während Sie diesen Satz lesen, werden etwa eine Million WhatsApp-Nachrichten verschickt. Wir teilen darin die Details unseres Privatlebens, von unseren größten Hoffnungen bis hin zu unseren tiefsten Ängsten. Das ist nur möglich weil wir glauben, dass sie auch privat bleiben.

Um die Privatsphäre von Nutzern zu schützen, implementieren wir eine Android App namens \gls{HiPS}. Sie schafft eine Schutzschicht, die in populären Instant Messengern derzeit fehlt: Steganografie. Wir führen \glspl{LLM} lokal auf Smartphones aus, um geheime Botschaften in unauffällige Cover-Texte zu kodieren. Unsere Arbeit basiert auf dem Stegasuras-Projekt. Durch die Portierung zu llama.cpp sind wir in der Lage, \glspl{LLM} auszutauschen um mit der verfügbaren Hardware zu skalieren. Wir erzeugen Chat-Konversationen zwischen Cover-Texten, steuern ihre Qualität über einen System Prompt und Klartext-Nachrichten, und verfeinern sie mit Emojis.

Schließlich führen wir eine Umfrage unter Studenten der TU Darmstadt durch, um unsere Cover-Texte mit echten Chat-Konversationen zu vergleichen. Dies liefert uns wertvolle Einblicke darin, wie wir die Qualität von Cover-Texten auf diese besonders relevante Nutzergruppe zuschneiden können.
\end{otherlanguage}

\endgroup

\vfill
