% !TeX root = ../Thesis.tex

%************************************************
\chapter{Introduction}\label{ch:introduction}
%************************************************
\glsresetall % Resets all acronyms to not used

Every day, we send tens of billions of messages on instant messengers like WhatsApp, Signal or Telegram~\cite{smithMillionWhatsAppMessages2018}. We share deeply personal information in them, such as our political beliefs, sexual orientation, religious views, or financial situation. Sharing what makes us vulnerable is only possible because we believe it to remain private. But as a result, the mobile devices we use to communicate are frequent targets of attacks. Once an attacker gets hold of our communication, they can use it against us in various ways. This ranges from stalking in abusive relationships to evidence in legal proceedings~\cite{obrienNebraskaTeenMother2022,mackeyFrenchScientistDenied2025}.

One might think that modern instant messengers are equipped with sufficient security measures to protect our data, given the widespread use of technologies like \gls{E2EE} or \gls{2FA}. But most of these mechanisms primarily address remote attackers, overlooking the threat of a physically present attacker. This could be the partner in a relationship or a law enforcement officer. Currently, protection against this type of attacker is limited. Typical measures include local authentication methods, such as unlocking the messenger app with a fingerprint, or self-destructing messages that delete after a set timer. But these approaches come with significant drawbacks. The former can be bypassed easily by the attacker physically forcing the victim to unlock the app. The latter often does not provide the desired user experience for everyday life.

Steganography presents a promising solution. It is an umbrella term for techniques that involve concealing information within an unsuspicious cover signal. Possible cover signals commonly include digital media like images, audio, or video, but can also be natural language text. Natural language in particular is beneficial as a cover signal because it is independent from any one communication medium~\cite{zieglerNeuralLinguisticSteganography2019}: Text can be sent digitally in chat messages or e-mails, but may also be spoken during a phone call, hand-written in a letter or printed in a newspaper. We leverage this to encode sensitive \textit{secret messages} into unsuspicious \textit{cover texts} that can be sent via any instant messenger~\cite{zieglerNeuralLinguisticSteganography2019}. As a result, even a physically present attacker could be misled, protecting the victim from potential harm. This approach introduces a layer of security currently missing in popular instant messengers.

By implementing the core functionality of a chat app with built-in steganography on Android, we set the precedent for further research and development. Our app is called \gls{HiPS}\footnote{Source code is available at \url{https://github.com/tobiasvonderheidt/hips}.}. It runs a \gls{LLM} locally on the smartphone to generate the cover text for a given secret message and vice versa. As the scope of this thesis doesn't involve a server backend, users of our app can try the steganography in a demo conversation view. Furthermore, we offer a standalone functionality to use steganography with existing instant messengers by copy-pasting messages. This enables real-world users to protect themselves already today.

We implement steganography with two algorithms demonstrated in a project called Stegasuras~\cite{zieglerNeuralLinguisticSteganography2019}. Its modular architecture allows for future adaptations to which algorithms are used in each of the following steps:

\begin{enumerate}
    \item Convert the secret message from string to binary, compressing it in the process.
    \item Encrypt the secret message bits (optional).
    \item Generate a cover text by completing a given context with the \gls{LLM}, thereby encoding the secret message bits.
\end{enumerate}

We extend the functionality of Stegasuras in multiple ways. First, we port it to the state-of-the-art llama.cpp framework~\cite{gerganovGgerganovLlamacpp2024} to run \glspl{LLM} locally. This gives us to access a broad range of \glspl{LLM} readily available on platforms like HuggingFace~\cite{huggingfaceModelsHuggingFace2025}. Adapting our implementation to different devices becomes as easy as swapping out the \gls{LLM}: We are able to deploy a small \gls{LLM} on an entry-level smartphone the same way we deploy a large \gls{LLM} on a flagship\footnote{Note that there is no clear definition when a language model is considered \textit{large}. For our purposes, any modern language model is called \gls{LLM}.}. Furthermore, we are able to create chat conversations of arbitrarily interleaved cover texts and plain texts. This also enables us to expose a system prompt to allow for easy low-level access to the \gls{LLM} via a natural language command. We enrich our cover texts with emojis to make them more organic. Overall, we create value and ensure longevity of our app by fulfilling the following requirements we set ourselves:

\begin{enumerate}
    \item Create a chat conversation between cover texts by generating them with a \gls{LLM}.
    \item Run the \gls{LLM} locally on an Android smartphone, i.e. don't require any internet connection for the steganography itself.
    \item Achieve acceptable performance on entry-level devices.
    \item Make the \gls{LLM} swappable.
\end{enumerate}

Lastly, we conduct a survey amongst students of TU Darmstadt to evaluate the plausibility of our cover texts in comparison to real chat conversations. This delivers profound insights into the linguistics of chat messages and serves as a starting point for further optimizations of cover text quality. By creating a well-documented, easy-to-maintain implementation of text-based steganography on Android, we open up many opportunities for further research to be conducted at SEEMOO.
