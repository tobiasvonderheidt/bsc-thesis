% !TeX root = ../Thesis.tex

%************************************************
\chapter{Implementation}\label{ch:implementation}
%************************************************
\glsresetall % Resets all acronyms to not used

In this chapter, we want to discuss the details of our implementation of text-based steganography on Android. First we will give a high-level overview of the app in \cref{sec:overviewOfTheApp}. Then \cref{sec:jni} will show an example of how Java/Kotlin and C++ code can interact using the \gls{JNI}. \cref{sec:tokenGenerationWithLlamaCpp} will introduce all necessary abstractions to understand token generation with llama.cpp. \cref{sec:algorithms} will explain the algorithms we used for steganography. So far, this is what was needed to port the functionality of Stegasuras~\cite{zieglerNeuralLinguisticSteganography2019} to Android.

We expand this functionality already in \cref{sec:algorithms} by implementing another compression algorithm. Then we will show how to improve cover text quality in \cref{sec:finishingTheLastSentence,sec:creatingAConversationBetweenCoverTexts,sec:emojis}: First we will show how to handle the last sentence, then how to create a conversation, and lastly how to generate emojis.

\section{Overview of the app}
\label{sec:overviewOfTheApp}

\subsection{Home screen}
\label{sec:homeScreen}

\subsection{Conversation screen}
\label{sec:conversationScreen}

\subsection{Settings screen}
\label{sec:settingsScreen}

\section{Java Native Interface}
\label{sec:jni}

\section{Token generation with llama.cpp}
\label{sec:tokenGenerationWithLlamaCpp}

\section{Algorithms}
\label{sec:algorithms}

\subsection{Binary conversion and compression}
\label{sec:binaryConversionAndCompression}

\subsubsection{Arithmetic compression}
\label{sec:arithmeticCompression}

\subsubsection{Huffman compression}
\label{sec:huffmanCompression}

\subsubsection{UTF-8}
\label{sec:utf8}

\subsection{Steganography encoding/decoding}
\label{sec:steganographyEncodingDecoding}

\subsubsection{Arithmetic coding}
\label{sec:arithmeticCoding}

\subsubsection{Huffman coding}
\label{sec:huffmanCoding}

\section{Finishing the last sentence}
\label{sec:finishingTheLastSentence}

\section{Creating a conversation between cover texts}
\label{sec:creatingAConversationBetweenCoverTexts}

\section{Emojis}
\label{sec:emojis}
