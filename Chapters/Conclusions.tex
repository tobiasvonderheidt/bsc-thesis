% !TeX root = ../Thesis.tex

%************************************************
\chapter{Conclusions}\label{ch:conclusions}
%************************************************
\glsresetall % Resets all acronyms to not used

By implementing \gls{HiPS}, we realize steganography in chat conversations using local \glspl{LLM} on Android. We achieve acceptable performance even on entry-level smartphones. While our app can be developed further to become a new instant messenger, it also offers a standalone functionality to work with existing messaging apps today. This improves privacy as it enables users to add another layer of protection to their communication.

To achieve this, we port the algorithms implemented in the Stegasuras project to the popular llama.cpp framework. This enables us to swap \glspl{LLM} to scale with available hardware. Our extensive refactoring and documentation make \gls{HiPS} easy to maintain. We extend Stegasuras by creating chat conversations between cover texts, exposing a system prompt for easy low-level access to the \gls{LLM}, allowing arbitrarily interleaved cover texts and plain texts, and enriching cover texts with emojis. Ultimately, we are able to fulfill all requirements we set ourselves in \cref{ch:introduction}.

Lastly, we conduct a survey amongst students of TU Darmstadt to compare our cover texts to real chat messages. This delivers valuable insights in how to fine-tune cover text quality for this highly relevant demographic of users.
