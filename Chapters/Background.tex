% !TeX root = ../Thesis.tex

%************************************************
\chapter{Background}\label{ch:background}
%************************************************
\glsresetall % Resets all acronyms to not used

To deepen our understanding for the underlying motivation of this thesis, we take a dive into the backgrounds of both steganography and \glspl{LLM}. Thereby, we deduct the requirements for our app.

\section{Steganography}
\label{sec:steganography}
To justify the relevance of steganography for the average user of instant messengers, we investigate its origins and involve some of today's politics. Then we formalize it and distinguish it from adjacent fields to lay the foundation our app is built upon.

\subsection{From physical to digital steganography}
\label{sec:fromPhysicalToDigitalSteganography}
The term "steganography" originates from ancient Greece and can be translated to "hidden writing"~\cite{kolataVeiledMessagesTerror2001,dembartEndUserHide2001}. Old Greek historian Herodotus documented it as early as 440 B.C.: Histiaeus, tyrant ruler of Miletus, was being held captive and wanted to send a message to his ally, Aristagoras. Since he had to bypass the guards of his enemies, the message had to be secret. He decided to shave the head of his most trusted slave, tattoo the message onto his scalp and wait for the hair to grow back before sending him on his way. With the message obscured, the slave would arrive at his destination, only for his hair to be shaved off again to read the secret message back~\cite{bennettLinguisticSteganographySurvey2004,petitcolasInformationHidingSurvey1999,dembartEndUserHide2001}.

This technique was still used by German spies until the early twentieth century~\cite{petitcolasInformationHidingSurvey1999}. Similarly, Nazi Germany used microdots in World War II: They shrunk pages of text down to the size of the dots in characters i or j, periods or commas, and sent them in printed letters. Invisible to the naked eye, this achieves an impressive information density~\cite{dembartEndUserHide2001,petitcolasInformationHidingSurvey1999}.

Today, we use digital data instead of physical objects to hide information~\cite{bennettLinguisticSteganographySurvey2004}. Digital steganography gained public attention as early as 2001, with Osama bin Laden encoding orders to members of his terrorist group Al-Qaeda as secret messages in digital files~\cite{dembartEndUserHide2001,schneierTerroristsSteganography2001}. This enabled them to use the public internet as communication channel, without needing to meet or even know each other~\cite{schneierTerroristsSteganography2001}. This shows that once a protocol is established, steganographic communication can be highly anonymous, asynchronous, and resilient~\cite{schneierTerroristsSteganography2001}.

\subsection{The problems of today}
\label{sec:theProblemsOfToday}
Contrary to what one might think after reading the previous paragraphs, steganography is not only interesting to tyrants, Nazis, terrorists and academics. Any internet user can benefit from protecting their privacy, as the commercialization of the internet has created complex power dynamics between governments, corporations and the public. Corporations store a lot of sensitive data about users to understand their behaviour, all in the name of improving their products~\cite{duportailAskedTinderMy2017,titcombMillionsPeoplesDNA2025}. This arouses interest by governments~\cite{greenwaldNSAPrismProgram2013}, who could utilize such data about their citizens to maintain political power. While it can be used to tackle challenges like the terrorists mentioned earlier, it can also be abused.

In the strive for power, surveillance is a proven tool. We cannot assume ourselves or our allies to be exceptions to this~\cite{macaskillGCHQTapsFibreoptic2013}. In recent years, many democracies have gone through debates about privacy as a human right versus surveillance in the name of security. Most prominently, this was ignited in 2013 by Edward Snowden's revelations about the global mass surveillance programs of the \gls{NSA}~\cite{greenwaldEdwardSnowdenWhistleblower2013}. More recent examples include the proposed "chat control" measures in the European Union~\cite{danielChatControlEnd2024} and the ban of Apple's Advanced Data Protection in the United Kingdom~\cite{kleinmanUKGovernmentDemands2025,kleinmanApplePullsData2025}. By governments forming public-private partnerships with software companies in the security sector, surveillance is being commercialized~\cite{bbcPegasusSpywareSold2021,kasterPrivatizedEspionageNSO2023}.

With the rise of political extremism in recent years, the risk of wars is present again. Traditional warfare has expanded into a new dimension, the cyberspace~\cite{serpanosCyberwarfareUkraine2022}. The ability to intercept an enemy's communication in real time can decide about life or death, democracy or dictatorship. For example, this practice is part of the Russian invasion of Ukraine~\cite{sufiSocialMediaAnalytics2023}.

To protect ourselves against these threats, we have to protect our communication. Steganography presents a promising solution. As it hides sensitive information in unsuspicious cover media, it can render our communication useless to attackers. Unfortunately, steganography is not commonly built into the tools we use for communication. By implementing an Android app to perform steganography in chat messages, we solve this problem for instant messengers.

\subsection{Formalizing steganography}
\label{sec:formalizingSteganography}
To be able to implement steganography, we have to formalize it. We do this by defining all relevant terms related to it. First, steganography itself is often described as both the art and science of hiding information~\cite{bennettLinguisticSteganographySurvey2004,wuGenerativeTextSteganography2024}. It can come in many forms, limited only by our creativity. We focus on digital steganography, i.e. on hiding data in other data. More specifically, we implement text-based or linguistic steganography, i.e. we work with natural language text~\cite{zieglerNeuralLinguisticSteganography2019}.

There are many reasons why one would want to communicate privately. All of them can be described by a common denominator: We have two parties, Alice and Bob. They communicate with each other by sending messages. A message is the data transmitted between them~\cite{wuGenerativeTextSteganography2024}. There also is Eve, an attacker who wants to compromise this communication~\cite{al-aniOverviewMainFundamentals2010,wuGenerativeTextSteganography2024}.

In text-based steganography, we say that we encode a secret message into a cover text, thereby hiding it. The secret message is something sensitive that Alice and Bob actually want to communicate. The cover text is something unsuspicious, so that Alice and Bob can exchange it as a message without arousing Eve's attention~\cite{al-aniOverviewMainFundamentals2010}. Alice encodes the secret message into the cover text before sending it to Bob, Bob decodes it back from the cover text after receiving it~\cite{al-aniOverviewMainFundamentals2010}.

To work with this, we need to define protection goals and threat models for our communication system. There are various protection goals commonly defined in information security. Confidentiality, integrity and availability (collectively known as the "CIA triad") are prominent examples~\cite{aliIoTSecurityReview2019}. But for our use case, we focus on the goal of privacy. Privacy can be defined as to ensure that data can only be accessed by authorized parties\footnote{Note that confidentiality can be defined similarly~\cite{chowdhuryChatGPTThreatCIA2023}.}~\cite{chowdhuryChatGPTThreatCIA2023}. The relevant data is what is hidden in the messages that Alice and Bob exchange.

Similarly, there are various aspects one can define about a threat model. Again, we focus on what makes our use case unique. We have a key assumption: The attacker can read all messages sent between Alice and Bob. This includes past, present and future messages. We also have a key restriction: The attacker should think that there is no secret communication happening~\cite{al-aniOverviewMainFundamentals2010}. They may know how the steganography works, but they don't know all parameters required to decode a message. This is similar to other common threat models (e.g. Dolev-Yao~\cite{dolevSecurityPublicKey1983}).

As steganography effectively hides the fact that there is secret communication happening, it can be seen as violating Kerckhoffs' principle: Kerckhoffs demands that a \textit{cryptographic} system should remain secure even if an attacker knows everything about the system, except the secret key~\cite{andersonLimitsSteganography1998,smithEffectiveSecurityObscurity2022} - or as Claude Shannon phrased it: "assume that the enemy knows the system"~\cite{shannonCommunicationTheorySecrecy1949}. The opposite is also called "security through obscurity"~\cite{smithEffectiveSecurityObscurity2022}. In steganography, we don't necessarily deal with secret keys. But the security of our \textit{steganographic} system partially relies on the attacker's unawareness of the steganography being used~\cite{al-aniOverviewMainFundamentals2010}. This is not an issue by itself, if steganography is used correctly: It is not supposed to be used \textit{in place} of other security mechanisms like authentication, access control and encryption, but \textit{in addition} to them~\cite{al-aniOverviewMainFundamentals2010}.

Steganography can also secure a communication channel that doesn't allow any other security mechanisms. This can be demonstrated especially well with text-based steganography, as it has a distinct advantage: It is independent from any one communication medium~\cite{zieglerNeuralLinguisticSteganography2019}. For example, a cover text can be sent digitally as a chat message or e-mail, but it can also be spoken during a phone call, hand-written in a letter or printed in a newspaper.

Lastly, an attacker can try to detect and break steganography by performing steganalysis~\cite{yangSeSyLinguisticSteganalysis2022}. Our implementation of text-based steganography should be robust against different steganalysis approaches. To maximize robustness, we deduct the following requirement:
\begin{itemize}
    \item Our implementation should create a chat conversation between cover texts by generating them with a \gls{LLM}. This is to make the cover texts as believable as possible.
\end{itemize}

Steganalysis can be done by humans or machines, depending on the criteria to be analyzed and the amount of data to be processed~\cite{zieglerNeuralLinguisticSteganography2019}. Due to the personal nature of chat messages, our steganalysis focusses on their unique linguistic characteristics (e.g. use of emojis) and is done by humans (see \cref{sec:qualitativeEvaluation}).

\subsection{Relation to cryptography}
\label{sec:relationToCryptography}
Both cryptography and steganography protect privacy by preventing attackers from accessing data, but they do so in fundamentally different ways~\cite{al-aniOverviewMainFundamentals2010,qadirReviewPaperCryptography2019}. Cryptography protects data by encrypting it with a secret key~\cite{qadirReviewPaperCryptography2019}. If Kerckhoffs' principle is followed, this renders the encrypted data useless without knowledge of the secret key~\cite{andersonLimitsSteganography1998,smithEffectiveSecurityObscurity2022}. But when an attacker encompasses encrypted data, it suggests to them that a secret communication is happening~\cite{malikHighCapacityText2017}.

Steganography uses a different approach. As a mechanism to hide data, it tries to deflect the attacker's attention elsewhere~\cite{al-aniOverviewMainFundamentals2010}. If steganography is performed \textit{after} encryption, the attacker may be able to read the cover texts, but they should think that there is no secret communication happening. Effectively, steganography tries to secure communication by creating a covert channel~\cite{gaureL^2M^2C^22024}. This fits steganography into the common scheme of cybersecurity being composed of layers~\cite{wilsonFundamentalCybersecurityConcepts2014,aliIoTSecurityReview2019,reegardConceptCybersecurityCulture2019}. The more layers an attacker has to get through, the more secure a system is.

\section{Large language models}
\label{sec:largeLanguageModels}
While the history of \gls{AI} chatbots goes back to the 1960s~\cite{weizenbaumELIZAComputerProgram1966}, they gained public attention only in recent years. Since the release of ChatGPT in late 2022, \gls{AI} chatbots have become popular for personal, educational and professional applications~\cite{wuUnveilingSecurityPrivacy2024}. This was enabled by the Transformer architecture introduced in 2017~\cite{vaswaniAttentionAllYou2023}, which is the basis for most \glspl{LLM} we use today.

As many applications of \gls{AI} chatbots require vast computing resources, most individuals and organizations rely on cloud-based service providers to host them. These service providers have interest in harvesting our personal data for commercial gain~\cite{liesenfeldOpeningChatGPTTracking2023}. For example, many free plans come with the condition that our prompts will be used as training data. This raises various privacy concerns. Malicious users could extract inputs of other users from the training data by prompting the \gls{LLM} with reverse psychology~\cite{guptaChatGPTThreatGPTImpact2023,wuUnveilingSecurityPrivacy2024}. Law enforcement could obtain access to user data without users being aware of it or having any legal recourse. This increases attack surface significantly.

Therefore, our implementation of text-based steganography should not rely on service providers to process secret messages. To minimize attack surface, we deduct the following requirements:
\begin{itemize}
    \item Our implementation should run the \gls{LLM} locally on an Android smartphone, i.e. it shouldn't require any internet connection for the steganography itself. This is to ensure privacy of our secret messages.
    \item Our implementation should achieve acceptable performance on entry-level devices. This is to ensure we can target a diverse user spectrum.
    \item Our implementation should make the \gls{LLM} swappable. This is to ensure longevity by being able to scale with available hardware.
\end{itemize}

By fulfilling these requirements, we make our app modular and minimize dependence on third parties. The last requirement in particular opens up many research opportunities, such as deploying a \gls{LLM} trained on our own chat messages to improve cover text quality~\cite{donnerSimulationMeFinetuning2024}. It also allows us to avoid many ethical concerns by choosing open source \glspl{LLM}~\cite{vandersarMetaTorrented812025,perrigoExclusive$2Hour2023}.
